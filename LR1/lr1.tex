\documentclass[12pt]{article}
\usepackage[a4paper, total={6in, 9in}]{geometry}
\usepackage{graphicx}
\graphicspath{ {./images/output/} }
\usepackage{caption}
\usepackage[english]{babel}
\usepackage{titling}
\usepackage{float}
% \usepackage{amsmath}
% \usepackage{minted}
% \usepackage{multicol}
% \usepackage{array}
% \usepackage{setspace}
% \usepackage{placeins}

% \usepackage{lipsum}

\title{Basics of Oscilloscope and Signal Generator}
\author{}
\date{}

\pagenumbering{gobble}
\begin{document}
\vspace*{\fill}
\begin{center}

    \emph{Heaven's Light is Our Guide} \\
    \textbf{Rajshahi University of Engineering and Technology} \\

    \begin{figure}[H]
        \centering
        \includegraphics[scale=.34]{images/RUET_logo.png}
        \label{fig:ruet_logo}
    \end{figure}
    \vspace{5mm}

    \textbf{Course Code}\\
    ECE 4144\\
    \vspace{3mm}
    \textbf{Course Title}\\
    Biomedical Engineering Sessional

    \vspace{5mm}
    \textbf{Experiment Date:} {},\\
    \textbf{Submission Date:} {}\\

    \vspace{5mm}
    \textbf{Lab Report 1: \\
        Basics of Oscilloscope and Signal Generator}

    \vspace{15mm}

    \begin{tabular}{c|c}
        \textbf{Submitted to} & \textbf{Submitted by} \\
        Md Mayenul Islam      & Md. Tajim An Noor     \\
        Assistant Professor   & Roll: 2010025         \\
        Dept of EEE, Ruet     &                       \\
    \end{tabular}

\end{center}
\vspace*{\fill}


\pagebreak

\tableofcontents

\pagebreak
\pagenumbering{arabic}
\maketitle

\section*{Theory}
\addcontentsline{toc}{section}{Theory}

% \begin{figure}[H]
%     \centering
%     \includegraphics[width=0.8\textwidth]{sigGen2.png}
%     \caption{Signal Generator}
%     % \label{fig:Signal Generator}
% \end{figure}


\section*{Discussion}
\addcontentsline{toc}{section}{Discussion}
In this experiment, we worked with two essential tools in electronics laboratories: the oscilloscope and the signal generator. The oscilloscope allows us to visualize electronic signals, while the signal generator produces a variety of signals at adjustable frequencies. Together, these devices play a critical role in testing and analyzing electronic circuits.\\\\
We gained hands-on experience by using these tools to observe and evaluate signals. Adjustments were made to both devices to see how changes in settings affected the signals. Additionally, we connected the signal generator to the oscilloscope to monitor the generated waveforms directly on the screen.\\\\
This practical session was highly beneficial, providing us with valuable experience in using these tools effectively. It demonstrated their importance in the field of electronics for testing, troubleshooting, and analyzing circuits.\\\\
In summary, the oscilloscope and signal generator are indispensable instruments in any electronics laboratory. The oscilloscope offers a detailed representation of signal characteristics, while the signal generator provides controlled test signals. By combining their functionalities, they allow engineers to create, measure, and analyze signals, ensuring that electronic devices and circuits perform as expected.

\section*{Precaution \& Conclusion}
\addcontentsline{toc}{section}{Precaution}
To ensure safe and effective use of the equipment, the following precautions were observed during the experiment:
\begin{itemize}
    \item Both the oscilloscope and signal generator were powered off when not in use to prevent potential damage.
    \item Probes and cables were connected or disconnected only when the devices were switched off to maintain safety and avoid electrical faults.
    \item Settings on the instruments were adjusted carefully to prevent any misconfigurations or damage to the devices.
    \item Only the provided, calibrated probes and cables were used to ensure accurate measurements.
    \item After completing the experiment, all equipment was returned to its designated safe storage location.\\\\
\end{itemize}
This experiment reinforced the critical role of the oscilloscope and signal generator in electronics. It highlighted their complementary functionalities, with the oscilloscope visualizing signal behavior and the signal generator providing precise, controllable inputs for testing. Through this exercise, we developed a deeper understanding of how these tools are used in real-world applications to diagnose, validate, and optimize electronic circuits. Moreover, adhering to proper precautions ensured the longevity of the equipment and the safety of the users, emphasizing the importance of best practices in a laboratory setting.

\bibliographystyle{IEEEtran}
\renewcommand{\bibname}{References}
\addcontentsline{toc}{section}{References}
\bibliography{ref}

\end{document}
